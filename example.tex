\documentclass{my_cv}
\usepackage[skins]{tcolorbox}

\begin{document}

\begin{multicols}{2}[
    \titletext{Thalys De Araujo Silva}%
        {}%
        {Estudante de Engenharia da Computação}%
        {thalysstark715@gmail.com}%
        {+55 98 98524-6367}%
        {github.com/thalysqda}%
        {linkedin.com/in/thalys-engenharia}%
]
\end{multicols}

\vspace{-1cm} % Ajuste para reduzir o espaço em branco após o nome

\section{\faFileText}{RESUMO}
Estudante de Engenharia da Computação no IFMA, com sólido interesse em desenvolvimento de software. Possuo experiência prática em projetos utilizando Python e Java para lógica de programação e backend. No front-end, desenvolvo interfaces utilizando HTML, CSS e JavaScript.

\begin{multicols}{2}
\section{\faBriefcase}{EXPERIÊNCIA}

\work{Estagiário de TI / Jan 2025 -- Dez 2025}%
    {Faculdade Unibras}%
    {Suporte técnico de hardware e software, manutenção de redes e auxílio no desenvolvimento de sistemas internos da instituição.}%
    {Suporte, Hardware, Redes, Windows Server}

\section{\faCode}{PROJETOS}

\work{Chatbot com Regras / 2025}%
    {Projeto Pessoal}%
    {Desenvolvimento de um assistente virtual utilizando fluxos lógicos e árvores de decisão para automação de atendimentos.}%
    {Python, Lógica, Automação}

\work{Reconhecimento Facial / Em andamento}%
    {Iniciação Científica / Acadêmico}%
    {Sistema de identificação facial utilizando o algoritmo LBPH (Local Binary Patterns Histograms) com a biblioteca OpenCV.}%
    {Python, OpenCV, LBPH, Visão Computacional}

\work{Projeto Web Full Stack / 2025}%
    {Estudos}%
    {Desenvolvimento de interfaces responsivas e dinâmicas utilizando tecnologias modernas de front-end.}%
    {HTML5, CSS3, JavaScript}
    
\section{\faList}{HABILIDADES}

\textbf{Linguagens:} Python, Java, JavaScript, HTML, CSS, SQL, C/C++

\noindent\textbf{Ferramentas:} OpenCV (LBPH), NumPy, Pandas, Git/GitHub, VS Code, Flask

\noindent\textbf{Outros:} Visão Computacional, Desenvolvimento Web Full Stack, Sistemas Operacionais, Redes de Computadores, Suporte Técnico

\columnbreak

\section{\faGraduationCap}{FORMAÇÃO ACADÊMICA}

\school{Bacharelado em Engenharia da Computação}%
{IFMA - Campus Santa Inês}%
{Previsão de conclusão: 202X} % Troque pelo seu ano de formatura

\section{\faCertificate}{CERTIFICAÇÕES E CURSOS}

\begin{itemize}[noitemsep]
    \item \textbf{Desenvolvimento Web:} JavaScript, HTML5 e CSS3.
    \item \textbf{Visão Computacional:} OpenCV e algoritmos de reconhecimento facial (LBPH).
    \item \textbf{Lógica de Programação:} Java e Python voltado a objetos.
\end{itemize}

\section{\faHeartbeat}{INTERESSES}

\begin{itemize}[noitemsep]
    \item Sistemas Embarcados e Internet das Coisas (IoT).
    \item Inteligência Artificial aplicada à segurança.
    \item Desenvolvimento de Software Open Source.
\end{itemize}

\end{multicols}
\end{document}